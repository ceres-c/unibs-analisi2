\begingroup % This group is used to alter exercises numbering scheme
\renewcommand\theexercise{\arabic{exercise}} % Use same numbering scheme as in publicly available exam sheets
\chapter{Temi Esame}
\section{T.E. 3 - AA 2019/2020 - 2020/04/24}
\begin{exercise}
	% TODO Ex 1
\end{exercise}
\begin{exercise}
	% TODO Ex 2
\end{exercise}
\begin{exercise}
	Si considerino in $\R^+$ le metriche $d(x, y) = \abs{y - x}$ e $D(x, y) = \abs{e^{-y^2} - e^{-x^2}}$.  Quale/i delle seguenti affermazioni è/sono certamente vera/e?
	\begin{enumerate}[noitemsep]
		\item $d$ e $D$ sono equivalenti.
		\item $(\R^+, D)$ non è completo.
	\end{enumerate}
	\begin{solution}~
		\begin{enumerate}
			\item
				Come prima cosa si nota che entrambe le metriche in oggetto sono illimitate, ma in modo "diverso". Infatti, limitando l'analisi a $x$ e scegliendo un $y$ "qualsiasi" (non tendente a nessuno dei due estremi di $\R^+$)
				\begin{itemize}
					\item La metrica $d$ tende a $+\infty$ per $x \to +\infty$
					\item La metrica $D$ tende a $+\infty$ per $x \to 0$
				\end{itemize}
				Da \fullref{def:metr_equiv}, si sa che per verificare l'equivalenza delle metriche deve valere
				\[\exists c,C \in \intervalopen{0}{+\infty}:\; \forall x,y \in X \quad c \cdot d(x,y) \leq D (x,y) \leq C \cdot d (x,y)\]
				Ma è impossibile trovare un $C \neq +\infty$ tale per cui la D sia minorata da $d$ nel caso in cui $x \to 0$, dunque \textbf{non son equivalenti}.
			\item
				A priori, non esiste un modo "standard" per verificare la completezza di un insieme in una data metrica. Se il caso in oggetto è sottoinsieme \textbf{Chiuso} ed utilizza la \textbf{stessa metrica} di un altro \textbf{caso noto} (come \fullref{ex:sp_metr_compl_e_non}), allora è possibile utilizzare la \fullref{prop:subset_compl_e_compl}.\\
				Se non si rientra nel caso specifico indicato qui sopra, spesso la risposta è "lo spazio metrico \textbf{non} è completo". È però necessario verificarlo individuando una successione di Cauchy che non ammetta limite nell'insieme stesso, negando la \fullref{def:completo}.

				La forma, $e^{-x^2}$ usata nella metrica $D$, tende ad $1$ per $x \to 0$, dunque viene spontaneo cercare una successione $x_n$ che tenda a $0$ per $n \to \infty$. La successione più evidente è, ovviamente, la $x_n = \frac{1}{n}$ che, si vede immediatamente, verifica la \fullref{def:succ_cau}:
				\[\forall \varepsilon > 0\quad \exists \nu \in \N:\; \forall n,m > \nu \text{ vale } D(x_n, x_m) < \varepsilon\]
				Verificando ora la convergenza a $0$ mediante \fullref{def:lim_succ}:
				\[
					\lim\limits_{n \to \infty} D(x_n, 0) =
					\lim\limits_{n \to \infty} \abs{
						e^{-0^2} -
						e^{-\frac{1}{n}^2}
					} =
					\lim\limits_{n \to \infty} \abs{
						\frac{1}{e^0} -
						\frac{1}{e^{\frac{1}{n}^2}}
					} =
					\abs{
						\frac{1}{1} -
						\frac{1}{e^{0}}
					} =
					\abs{1 - 1} = 0
				\]
				Ma $0 \in \R \notin \R^+$, per definizione di $\R^+ = \intervalopen{0}{+\infty}$. Si è dunque individuato un controesempio alla \fullref{def:completo}, \textbf{negando la completezza}.
		\end{enumerate}
	\end{solution}
\end{exercise}

\section{T.E. 2012/2013 scritto n.1}
\begin{exercise}
	Sia $f: \R^2 \to \R$ data da $f(x, y)=e^{-\abs{4\cdot arctan(x\cdot y^2)}}$
	\begin{description}
		\item[A] Nessuna delle altre affermazioni è esatta
		\item[B] $f$ ammette almeno un punto di minimo assoluto
		\item[C] $\inf_R^2f = 0$
		\item[D] $f$ ha infiniti punti di massimo
	\end{description}
	L'esponenziale è una funzione monotona crescente quindi la ricerca di massimi a minimi si sposta alla ricerca dei massimi e minimi dell'esponente.\\
	L'esponente assume sempre valori negativi. Inoltre risulta essere una quantità limitata tra $[0;4\frac{\pi}{0}[$, quindi $\sup_R^2f=e^0=1$ e $\inf_R^2f=e^{-2\pi}$\\
	Sono quindi punti di massimo tutti i punti che rendono nullo l'esponente: $arctan(xy^2)=0 \implies x=0,\forall y or y=0,\forall x$ che sono i due assi. Essendo questi punti del dominio allora si può dire $\sup_R^2f=\max_R^2f=0$\\
	I punti di minimo si hanno per $\abs{arctan(xy^2)}=\frac{\pi}{2}$ quindi per $x \to \pm\infty$ or $y \to \pm\infty$ essendo questi valori al limite il valore $e^{-2\pi}$ è $inf$ per $f$\\
	La risposta vera è quindi la D.\\
\end{exercise}
\begin{exercise}
	Sia $(X,d)$ uno spazio metrico e siano $A, B$ sottoinsiemi di $X$. Quale/i delle seguenti affermazioni è/sono certamente vera/e?
	\begin{description}
		\item[1] $A\subseteq B \implies\partial A\subseteq \partial B$
		\item[2] $A\subseteq B \implies\overline{A}\subseteq\overline{B}$
	\end{description}
	\begin{description}
		\item[A] Entrambe
		\item[B] Solo la seconda
		\item[C] Nessuna delle affermazioni è esatta
		\item[D] Solo la prima
	\end{description}
	La prima affermazione è certamente falsa poiché se scelto come spazio metrico $\R^2$ con distanza quella euclidea. Scelgo $A=B((0,0),2), A=B((0,0),1)$ allora si ha che $\partial A = \{(x,y)\in \R^2:d((x,y),(0,0))=2\}$ e $\partial B = \{(x,y)\in \R^2:d((x,y),(0,0))=1\}$ e questi due insiemi sono disgiunti.\\
	la seconda è vera ma devo pensarci un po...\\
\end{exercise}


\endgroup % Close group and restore default exercise numbering scheme
